\documentclass{article}
\setlength{\oddsidemargin}{0in}
\setlength{\textwidth}{6.5in}
\setlength{\topmargin}{-.5in}
\setlength{\textheight}{9in}
\title{CARMEN Param Server}
\author{Nicholas Roy and Michael Montemerlo and Sebastian Thrun}

\begin{document}

\maketitle

\section{Introduction}

CARMEN programs use the \verb!param_server!, a repository or registry of
values for parameters to be used during the operation of a robot. The
intention of the parameter server is to ensure that all modules are using the
same parameters definitions, such as maximum allowable velocity, robot size,
etc. Many robot code suites require that each separate process reads the
parameter values directly from a file. We have chosen the parameter server
approach because different processes frequently reside on different computer,
which would entail having copies of the same file on each such computer. This
often leads to having copies of \emph{different} files on each computer, with
undesirable results. Secondly, the parameter server facilitates for dynamic
updating of parameter values; for instance, the maximum speed of the robot can
be changed during the operation of the robot, without restarting any
processes. This necessitates some additional overhead on the part of each
process, to ensure that each process is subscribed to changes in each
variable. This document will describe how best to do this.

The parameter server is intended to serve parameters that can vary across
robots, or across execution runs. Parameters that cannot change should not be
served by the parameter server. For example, the maximum preferred
translational velocity for a particular deployment is served as
\verb!robot_max_t_vel!. However, each robot (such as the scout) has a
practical maximum wheel velocity; this value should be hard-coded in the base
module for that robot, instead of being served. Similarly, for rflex robots,
the odometry conversion factor from wheel tics to metres is hard-coded, as
this is a hardware dependent value and cannot change. 

Finally, the parameter server can also be used as a map server. This ensures
that all processes are using the same map, with the same resolution, etc..
The does entail a performance penalty in terms of bandwidth loss at the start
up of each process, but we believe that resultant the ease of use, in terms of
needing only one copy of the map and ensuring consistency, is an acceptable
bargain.

\section{The Parameter Server}

\begin{verbatim}
Usage: param_server -robot <robot name> [map file] [ini file]
\end{verbatim}

The parameter server reads parameter definitions from the ini file, or, if no
ini file name is provided on the command line, then attempts to load
parameters from \verb!./carmen.ini!, \verb!../carmen.ini!, and finally
\verb!../src/carmen.ini!. If no parameter file can be found, then the process
exits with an error. 

To make life easier, multiple robot definitions can be contained in a single
ini file. Therefore, the parameter server requires a robot name to select a
set of parameters, using the required argument \verb!-robot <robot name>!. At
start up, the parameter server loads all the parameter definitions listed
under the generic [*] parameter set, and also all the parameters listed under
the matching robot name. 

The parameter server can also serve maps, subsuming the functionality of the
map server. 

Finally, the parameter server also performs sanity checks, to make sure that
all running CARMEN processes are the same version, to ensure consistency among
message formats, etc. All programs are expected to call
\verb!carmen_param_check_version()! immediately after initializing the IPC
connection. 

\section{The ini File Format}

The ini file format is ASCII, intended to be human readable and editable using
emacs. Given the somewhat hierarchical nature of the file, a markup language
such as XML may have been preferable, but this seems like overkill for a
fairly simple IO task.

The file consists of robot definition sections, delimited at the start by
\begin{verbatim}
[<section name>]
\end{verbatim}
and delimited at the end by the start of another section, or by the
end-of-file. The section name should either be \verb!*! (for all robots) or be
the name of a robot to be specified on the command line when starting the
parameter server.

Each section consists of parameter definitions, given by:
\begin{verbatim}
<module name>_<parameter name> <value>
\end{verbatim}

The module name can contain any characters except white space and the
underscore (`\verb!_!') character. The parameter name can contain any
characters except whitespace. The parameter value can contain any characters
at all, but the value ends with the end of line. There is no mechanism for
writing parameter values across multiple lines. Trailing whitespace is trimmed
from parameter values. 

Comment lines must have `\verb!#!' at the first character of each line. 

Parameter values can be no longer than 2048 characters, and the concatenation
of module names and parameter names can also be no longer than 255
characters. There is a limit of 128 unique module names.

If a variable is specified twice, then only the last value is retained.

\subsubsection*{A Very Short Example ini File}

\begin{verbatim}
[*]
navigator_update_map on

[scout]
base_type         scout
scout_dev         /dev/ttyS0
robot_width       0.39
robot_rectangular 0

[pioneer]
base_type         pioneer
pioneer_dev       /dev/ttyS0
pioneer_vesion    1
robot_rectangular 1
robot_width       0.48
robot_length      0.54
\end{verbatim}

\subsection{Specifying Parameters from the Command Line}

Parameter values can be temporarily over-ridden from the command line of a
given process, \emph{for that process only}. For example:
\begin{verbatim}
% robot -max_t_vel 0.1
\end{verbatim}
will specify a \verb!max_t_vel! of 0.1 m/s for the \verb!robot! process only. 

This parameter specification method is not advised, is a convenience only, and
may be removed from future versions of carmen. Command-line parameters are
assumed to have no module name, and attempts are made to find a parameter with
any module name and matching parameter name. If a process uses multiple
parameters with different module names and the same parameter name, the
behaviour of command-line specification is undefined.

\end{document}

 
