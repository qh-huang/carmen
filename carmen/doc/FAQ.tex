\documentclass{article}
\setlength{\oddsidemargin}{0in}
\setlength{\textwidth}{6.5in}
\setlength{\topmargin}{-.5in}
\setlength{\textheight}{9in}
\title{CARMEN Frequently Asked Questions}
\author{Nicholas Roy and Michael Montemerlo and Sebastian Thrun}

\begin{document}

\maketitle

\begin{enumerate}

\item What are the units of each parameter?
  
  CARMEN uses SI units internally. All distances are in metres. All angles are
  in radians. All velocities are in metres/second. All parameters described in
  an .ini file should obey this constraint, and any that do not should be
  considered a bug. 

  The only notable exception is that \verb!carmen_map_point_t! points are in
  map grid cells, and should be multiplied by the map resolution (or converted
  using \verb!carmen_map_to_world! to get distances in metres. 

\item How do I reduce the processor load of CARMEN?

  The major CPU load from CARMEN comes from two sources: localize and
  navigator. Bear in mind that reducing the CPU load of these two processes
  will lead to worse performance, and that a preferable scenario would be to
  move these processes off the robot to a more powerful CPU. 

  The navigator CPU usage can be reduced by changing the number of lasers used
  \\
  (\verb!navigator_num_lasers!), increasing the replanning frequency
  (\verb!navigator_replan_frequency!) or, for static worlds with very accurate
  maps and very good localization, dynamic obstacle avoidance can be turned
  off using \\ (\verb!navigator_update_map!). This last change is only
  recommended in carefully controlled environments.

  Localize CPU usage can be reduced by making sure people tracking is off
  (\verb!localize_people_tracking!) and reducing the number of particles
  (\verb!localize_robot_particles!). 

\item Do you support sonar sensors?

  Nominally, yes. However, the sonar support has not been tested on most
  platforms, and therefore should be considered in alpha development. Also,
  localization, navigation and map building are not in any way supported using
  sonars yet. This may change, but it is currently a low-priority item.

\item I want to make the robot move backwards, but it won't go.  

  There are two relevant parameters: \verb!robot_collision_avoidance! and
  \verb!robot_allow_rear_motion!.  
  
  If \verb!robot_allow_rear_motion! is off, then the robot will not move
  backwards.

  If you do not have backwards pointing sensors, such as a laser or sonar
  sensors, then the robot will not move backwards while collision avoidance is
  turned on. If you do have backwards pointing sensors, are they enabled? If
  no rear sensor data shows up in robotgraph, then the robot will not move
  backwards while collision avoidance is on.
  
  If you turn \verb!robot_allow_rear_motion! on, and
  \verb!robot_collision_avoidance! off, then the robot will always move
  backwards if you ask it to. But, there is no guarantee you won't hit
  anything. We do {\bf not} in any way endorse ever turning off collision
  avoidance. You {\bf will} hit things.

\item How can I run multi-robot simulations?

The way to run multi-robot simulations is to run multiple, parallel instances
of CARMEN, and have the simulators communicate with each other. 

You can allow a running simulator to ``see'' a second simulator using
\verb!void carmen_simulator_connect_robots(char *other_central)!. For example,
start a CARMEN simulation (central, param\_server, simulator, localize,
navigator, navigator\_panel). Start a second CARMEN simulation on a different
port, such as 1382. You can do this by starting central using
\begin{verbatim}
% central -p1382
\end{verbatim}
and then start all remaining programs (param\_server, simulator, localize,
navigator, navigator\_panel) with the \verb!CENTRALHOST! environment variable
set to \verb!machine:1382!. Make sure that you start both simulations using
the same map. 

In a third program, you need to connect the two simulations. For example, a
program like this would do the trick to tell the first simulator about the
second:

\begin{verbatim}
#include <carmen/carmen.h>

int main(int argc, char *argv[]) 
{
  carmen_initialize_ipc(argv[0]);
  carmen_param_check_version(argv[0]);

  carmen_simulator_connect_robots("localhost:1382");

  return 0;
}
\end{verbatim}

Most people would want both simulators to see each other, in which case you
would probably want a short program like this:

\begin{verbatim}
#include <carmen/carmen.h>

int main(int argc, char *argv[]) 
{
  /* Connect to the first IPC context, assumed to be running on
     localhost */

  carmen_initialize_ipc(argv[0]);
  carmen_param_check_version(argv[0]);

  /* Tell the first simulator about the second */ 

  carmen_simulator_connect_robots("localhost:1382");

  /* Connect to the second IPC context, assumed to be running on
     localhost:1382 */

  IPC_connectModule(argv[0], "localhost:1382");

  /* Tell the second simulator about the first. */ 

  carmen_simulator_connect_robots("localhost");

  return 0;
}
\end{verbatim}


Position the first robot (using the first \verb!navigator_panel!). Position
the second robot (using the second \verb!navigator_panel!) so that it is a
little bit in front of where you put the first robot (which of course you
can't see in the second \verb!navigator_panel!) and pointing back at the first
robot. If you start two instances of \verb!robotgraph!, you should see the
other robot in each robotgraph. As you drive the robots around, they should
interfere with each other and see each other, just as if they were physically
there. 

\item How can I control multiple robots in a multi-robot simulation?

The first thing is to get both simulations working, as described above.

The next thing is to understand multiple IPC contexts, reading the IPC
manual. 

When you subscribe to (most) IPC messages, you set up a handler (and maybe
message pointer) on a context-specific basis (although you only get one
handler and message allocation per context). 

So, if you wanted to navigation status messages from two simulations, you
could use the following:
\begin{verbatim}
void connect_to_two_simulations(char *simulation1, char *simulation2,
                                IPC_CONTEXT_PTR *context1,
                                IPC_CONTEXT_PTR *context2) 
{

  *context1 = IPC_connectModule(module_name, simulation1);
  carmen_navigator_subscribe_status(NULL, handler1, CARMEN_SUBSCRIBE_NOW);

  *context2 = IPC_connectModule(module_name, simulation2);
  carmen_navigator_subscribe_status(NULL, handler2, CARMEN_SUBSCRIBE_NOW);

  IPC_setContext(*context1);
  carmen_navigator_set_goal_place("lab door");
  carmen_navigator_go();

  while (1) {
    IPC_setContext(*context1);
    sleep_ipc(1);
    IPC_setContext(*context2);
    sleep_ipc(1);
  } 
}
\end{verbatim} 

What this does is establish a connection with a simulation running on the
\verb!CENTRALHOST! described by simulation1, and a simulation running on
\verb!CENTRALHOST! described by simulation2. (The simulations presumably were
connected earlier.) The function subscribes to navigator status messages from
each simulation, using different handlers for each message. You could
certainly handle messages from both simulations using the same function, using
the \verb!IPC_CONTEXT_PTR IPC_getContext();! function inside the handler to
tell which simulation generated the callback. 

After the callbacks are established, the function goes back to the IPC context
of the first simulation, sets a goal and starts the robot moving. After that,
the function indefinitely just handles messages.

\end{enumerate}

\end{document}
